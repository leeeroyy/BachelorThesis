\begin{frame}
    \frametitle{Current-voltage Characteristic}
    \only<1>{
        \begin{itemize}
            \item graphical representation of the relationship between the current
            \(I\) and voltage \(V\) of an electrical device
            \item often abbreviated as I-V characteristic or I-V curve
        \end{itemize}
        \begin{figure}
            \centering
            \includegraphics[scale=0.215]{IV_characteristic_at_STC.png}
        \end{figure}
    }
    \only<2>{  
        \begin{itemize}
            \item all except the ideal model are implicit, meaning they cannot be solved
            explicitly for \(I\) or \(V\) \\
            \(\implies\) I-V curve can be interpreted as a
            zero level set \(f(V, I) = 0\) contained in the first quadrant
            \item only a single point from the simulated I-V characteristic is needed to
            formulate a prediction
            \item this point ideally corresponds to the operating point
        \end{itemize}
    }
\end{frame}