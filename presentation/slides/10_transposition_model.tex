\begin{frame}
    \frametitle{Transposition Models}
    \only{
        Transposition models estimate the in-plane irradiance from global and diffuse
        horizontal irradiance. They have the general form:\footnote{The subscript \say{h}
        indicates horizontal irradiance, while \say{c} (collector) indicates in-plane irradiance.}
        \begin{align}
            G_{\text{c}} &= B_{\text{c}} + D_{\text{c}} + R_{\text{c}}
            \\[8pt]
            B_{\text{c}} &= B_{\text{h}} \, \frac{\cos \nu}{\cos \psi_{\text{s}}}
            \\[0pt]
            R_{\text{c}} &= G_{\text{h}} \, \tau_{R} \, \rho = G_{\text{h}} \, \frac{1 - \cos \beta}{2} \, \rho
             \\[8pt]
            D_{\text{\text{c}}} &= D_{\text{h}} \, \tau_{D}
        \end{align}
        where \(\rho\) is the surface albedo, and \(\tau_{R}\) and \(\tau_{D}\) are the
        transposition factors for the beam and diffuse components, respectively.
    }<1>
    \only{
        The angles in the transposition model are required to correctly transpose
        horizontal irradiance to the plane of the module.
        \begin{figure}
            \centering
            \includegraphics[scale=0.11]{images/Solar_Geometry.png}
            \caption{\small Visualization of angles used in transposition models.}
        \end{figure}
    }<2>    
    \only{
        \textbf{Surface orientation:}
        \begin{itemize}
            \item \(\alpha\): surface azimuth angle \hfill [\si{\degree}]
            \item \(\beta\): surface tilt \hfill [\si{\degree}]
        \end{itemize}
        \textbf{Sun position:}
        \begin{itemize}
            \item \(\alpha_{\text{s}}\): solar azimuth angle \hfill [\si{\degree}]
            \item \(\gamma_{\text{s}}\): solar altitude angle \hfill [\si{\degree}]
            \item \(\psi_{\text{s}}\): solar zenith angle \hfill [\si{\degree}] % (\(\psi_{\text{s}} = 90 \si{\degree} - \gamma_{\text{s}}\)) 
        \end{itemize}
        \textbf{Additional angles:}
        \begin{itemize}
            \item \(\nu\) is the surface incidence angle \hfill [\si{\degree}]
            \item \(\alpha_{\text{f}}\) is the wall-solar azimuth angle \hfill [\si{\degree}] % (\(\alpha_{\text{f}} = \alpha_{\text{s}} - \alpha \)) 
        \end{itemize}
    }<3>
\end{frame}