

\begin{abstract}
    Die vorliegende Arbeit bietet einen Einblick in die Modellierung
    von Photovoltaikanlagen mit dem Ziel, die erzeugte Gleichstrom-Leistung der Anlage
    für gegebene Umweltbedingungen zu simulieren. Zu diesem Zweck wird ein
    bekanntes physikalisches Modell verwendet, das die Strom-Spannungskurve
    eines einzelnen Photovoltaikmoduls in Abhängigkeit von der einfallenden
    Solarstrahlung und der Zelltemperatur beschreiben kann. Mithilfe
    praktischer Überlegungen lässt sich dieses Modell auf die Modellierung
    von Photovoltaiksystemen, bestehend aus mehreren Modulen, übertragen.
    Die benötigten Umweltparameter des Modells werden mit Hilfe von zwei
    weiteren bekannten Modellen, sogenannten Thermo- und Transpositionsmodellen,
    aus meteorologischen Daten abgeleitet. Dabei spielt das Wissen über
    die Modellierung der Solarstrahlung auf geneigte Flächen eine zentrale
    Rolle, worauf zu Beginn der Arbeit eingegangen wird.
    Des Weiteren werden alle Gleichungen und Berechnungsmethoden, die bei der
    Bestimmung der Modellparameter zum Einsatz kommen, umfassend beschrieben, sodass
    das Modell auf verschiedene Arten von Photovoltaikmodulen angewendet werden
    kann. Abschließend wird das Modell genutzt, um die Stromproduktion einer
    4,62 kWh Photovoltaikanlage in Bayern, Deutschland, zu simulieren.
    Die benötigten Wetterdaten wurden von nahegelegenen Wetterstationen des
    Deutschen Wetterdienstes bezogen. Die Analyse der Ergebnisse zeigt,
    dass die Genauigkeit des Modells erheblich von der Qualität der verwendeten
    Wetterdaten abhängt. In Zeiträumen, in denen die Abweichung zwischen den
    gemessenen und tatsächlichen Umweltbedingungen mit großer Sicherheit gering ist,
    zeigte das Modell eine hohe Genauigkeit.
\end{abstract}
