In this thesis, an approach to model the direct current (DC) power
output of a photovoltaic (PV) system was presented. The core of
the model is based on a well-known single-diode five-parameter
model that can be calibrated with a minimal set of data provided
by manufacturers. The calibration method allows for the determination
of the electrical parameters of a PV panel and the assessment of
the power generated under any conditions of operating temperature
and solar irradiance. This approach was extended to arbitrarily
configured and sized PV systems, taking into account practical
aspects of PV system design. Furthermore, all equations and calculations
involved in the model parameter determination process were presented
in a way that allows the model to be applied effectively to various
types of PV systems. The application of the model requires the
preliminary estimation of the operating conditions. To this end,
well-known thermal and transposition models were introduced,
capable of estimating the operating conditions from meteorological
data, including temperature, wind speed, air pressure, and global
and diffuse horizontal irradiance. The fundamentals related to
solar geometry and solar radiation modeling were introduced to
the extent necessary to understand the individual components of
the presented procedure.

The model was applied to a real-world PV system, and its performance
was evaluated over 154 days within a period of 166 days. The results
suggest that the model's accuracy is highly dependent on the quality
of the input data. The reliability of the model was confirmed during
time periods when the quality of the input data was expected to be
high, specifically, during clear-sky days when the PV system was
not shaded. These time periods were also used to estimate the PV
array's tilt angle, a required model parameter that may be unknown
in real-world applications, with a precision of 2 degrees. Conversely,
it was observed that lower-quality input data inevitably resulted in
reduced accuracy of the model's predictions. In the considered case,
the frequent presence of clouds during the evaluation period, combined
with the distance between the PV system and the irradiance measurement
instrument, as well as shading of the PV system by large trees during
the morning hours, contributed to discrepancies between the measured
and actual meteorological conditions, resulting in lower accuracy of
the model's predictions.

In conclusion, this thesis provided a comprehensive overview of
physical PV system modeling. The presented model can be successfully
used for energy assessments of PV systems that are exposed to stable
and primarily sunny weather conditions without shading obstructions.
Future work could focus on integrating shading analysis and higher-quality
environmental data to enhance the model's predictive capabilities. By
addressing the challenges associated with input data quality and
environmental factors, this thesis contributes valuable insights
to the field of solar energy modeling and aids in the optimization
and planning of PV installations.
