
The impact of human activities on Earth's climate has been unequivocally demonstrated
by scientists over the past decade. In response, global leaders have adopted common 
environmental goals to protect the planet. Consequently, the global energy production mix
has shifted significantly towards green energy. Given the inherently complex nature of
renewables, extensive research across various disciplines has been conducted to improve and 
understand emerging technologies. Solar energy stands out as one of the most important energy
resources for humanity, being clean, pollution-free and inexhaustible. Remarkably, the Earth
receives enough solar energy in a matter of hours to meet the world's energy demand of an
entire year. Advances in photovoltaic (PV) technology and reductions in manufacturing costs
have led to a multifold increase in PV capacity, often characterized by decentralized deployment
and fluctuating power output due to variable weather conditions.

Accurate modeling of PV modules is of vital importance in many fields.
It enables designers to optimize system performance and maximize cost-effectiveness.
Power forecasts support grid operators in balancing power demand and supply, while
plant operators utilize predictions for energy management and to achieve the best economic
outcome. The challenges posed by volatile production are being addressed by battery
systems, whose optimized charging management depends on reliable generation data.

The performance of a PV module largely depends on the availability of solar 
radiation and its conversion efficiency. These aspects are influenced by numerous
physical parameters, including the geographical location of the
site, panel orientation, weather conditions, surrounding obstructions,
the electrical load etc. \cite[p. 1358f]{LoBrano}. Unsurprisingly, energy assessments
based on simplistic models that assume a constant value for the conversion
efficiency can yield erroneously optimistic predictions \cite{Wang2021}.
Unfortunately, the information provided by the manufacturers often lacks the detail
needed to exploit high-performance predictive tools. An analysis of information
from over 400 manufacturers' websites revealed that the quality of this information is variable,
sometimes inconsistent, and often inadequate for reliable designs \cite[p. 1161]{Orioli}.

The variability of available manufacturing, meteorological and site-related data
resulted in a wide spectrum of forecasting models, broadly classified
into three main categories: physical, statistical and hybrid \cite{Ulbricht2013, Iheanetu2022}. 
Physical models simulate the conversion of solar irradiation into electricity
using physical equations. The more complex spectrum of these models can accurately describe
the behavior of PV modules when high-quality input and calibration data are available.
In contrast, statistical models aim to establish the relationship between input
variables and the corresponding power output without relying on physical equations.
Many statistical models require prior training on historical meteorological
and generation datasets, where training involves calculating model parameters
through mathematical optimization methods that accurately represent the
relationship between the datasets and can generalize to unseen data.
As might be expected, physical and statistical models can be combined to
create hybrid models. Such combinations may overcome individual drawbacks
and further improve accuracy, albeit with the potential downside of increased
complexity and greater computational resource requirements.

This thesis introduces an approach to predict the direct current
(DC) power output of a PV system of arbitrary size and configuration under
certain assumptions. A well-known five-parameter physical model
builds the basis of this approach, which can be calibrated with a
minimal set of data provided by manufacturers. Required meteorological
parameters include temperature, wind speed, air pressure as well as global
and diffuse horizontal radiation. The model also requires the geographical
location of the PV system and the orientation of the individual PV arrays in
order to estimate the input variables for the physical model. The 
presented approach can be combined with a conversion model to predict
the generated alternating current (AC) power required for load consumption
and connection to the electrical grid.

This thesis is structured as follows: Chapter \ref{sec:Solar Geometry and Radiation}
introduces the basics of solar geometry and solar radiation. Chapter 
\ref{sec:Photovoltaic system modeling and power prediction} details
the physical modeling of PV systems and introduces the five-parameter model.
In Chapter \ref{sec:Application and analysis of the model}, the presented model is
applied to a 4.62 kW PV system and analyzed. Finally, Chapter \ref{sec:Conclusion}
concludes the thesis and outlines potential directions for future work.
