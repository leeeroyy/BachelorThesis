
As mentioned at the beginning of this chapter, the mathematical models
presented in \ref{sec:Background to physical models} can
be used to analytically describe the I-V characteristic curve
of photovoltaic (PV) cells, modules, strings of modules, and arrays.
The following definition of the configuration of an array allows
the application of the single-diode five-parameter model to an array
of PV modules.

\begin{itemize}
    \item \(N_\text{c}\): Number of cells connected in series in each module.
    \item \(N_\text{m}\): Number of modules connected in series in each string.
    \item \(N_\text{s}\): Total number of cells connected in series in each string, i.e. \(N_\text{c} N_\text{m}\).
    \item \(N_\text{p}\): Number of strings connected in parallel in the array.
\end{itemize}

\noindent
With these definitions, the governing equation of the five-parameter
single-diode model can be adjusted to model an array of PV modules
\cite{Ma2014_2, Tian2012}:

\begin{align}
    I = \; &\alpha_{\text{G}}N_\text{p}I_{\text{L}}(T) - N_\text{p}I_{0}(\alpha_{\text{G}}, T)\Biggl[\exp\biggl(\frac{\alpha_{\text{G}}(V + KI(T - T_{\text{ref}}))+I\frac{N_\text{s}}{N_\text{p}}R_{\text{s}}}{\alpha_{\text{G}}N_\text{s}nT}\biggr) - 1\Biggr] \nonumber \\
                       &- \frac{\alpha_{\text{G}}(V + KI(T - T_{\text{ref}}))+I\frac{N_\text{s}}{N_\text{p}}R_{\text{s}}}{\frac{N_\text{s}}{N_\text{p}}R_{\text{sh}}}
    \label{eq:IV_curve_equation_for_arrays}
\end{align}

\noindent
The procedure presented in the last section is designed to accurately
model the current-voltage characteristic of a single PV module.
It utilizes the slopes of the I-V curve of a single module at
STC in the short-circuit and open-circuit points to fit the model
precisely at STC. To use Equation \ref{eq:IV_curve_equation_for_arrays}
to predict the current-voltage characteristic of an array of PV modules,
the parameters for a single module must first be determined
using Equation \ref{eq:Orioli_IV_curve_governing_equation}
following the described procedure. These parameters are then
substituted into Equation \ref{eq:IV_curve_equation_for_arrays}
with \(N_\text{s} = N_\text{m}\) and \(N_\text{p}\) as defined above. This adjustment
accounts for the fact that the parameters have been determined
to fit an entire module consisting of \(N_\text{c}\) series-connected cells.

In practice, a PV array may consist of several strings of modules
that are oriented in different directions. Each string receives
varying amounts of irradiance depending on its orientation,
resulting in distinct I-V curves for each string. Combining
these strings into a single array without accounting for
orientation differences can make it difficult for the inverter
to locate the global maximum power point (MPP). To address
this issue, modern PV systems typically connect strings facing
the same direction to a dedicated inverter equipped with maximum
power point tracking (MPPT) technology. MPPT dynamically adjusts
the operating point of each string to its individual MPP by
altering the impedance seen by the inverter, ensuring optimal
performance \cite[p. 152f]{Mayfield}. Therefore, when modeling
a PV array, it is both valid and practical to model groups of
modules that are oriented in the same direction and connected
to an inverter with MPPT separately. This modeling approach
assumes that the operating point of each string corresponds
to its individual maximum power point, and the total system
output is obtained by summing the contributions of all separately
modeled strings. This reflects real-world practices aimed at
maximizing a PV system's output. By modeling each string
individually based on its orientation, one can accurately
simulate the overall performance of PV systems or plants
consisting of multiple arrays, while accounting for practical
considerations in PV system design.

